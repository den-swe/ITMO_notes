% ================= ОФОРМЛЕНИЕ РАЗДЕЛОВ =================
\sectionfont{\color{blue}}
\subsectionfont{\color{blue}}
\subsubsectionfont{\color{blue}}

% ================= НУМЕРАЦИЯ РАЗДЕЛОВ =================
\renewcommand{\thesection}{\arabic{section}}
\renewcommand{\thesubsection}{\thesection.\arabic{subsection}}
\renewcommand{\thesubsubsection}{\thesubsection.\arabic{subsubsection}}

% ================= СТИЛИ ТЕОРЕМ =================
\declaretheoremstyle[
  headfont = \normalfont\bfseries,
  notefont = \normalfont\bfseries,
  bodyfont = \normalfont,
  headpunct = {},
  headformat={\NAME~\NUMBER.\NOTE},
]{theoremstyle}

\declaretheoremstyle[
  spaceabove = 6pt,
  spacebelow = 6pt,
  headfont = \normalfont\itshape,
  notefont = \normalfont,
  bodyfont = \normalfont,
  postheadspace = 3pt,
]{remarkstyle}

\declaretheoremstyle[
  spaceabove = 6pt,
  spacebelow = 6pt,
  headfont = \normalfont\bfseries,
  notefont = \normalfont,
  bodyfont = \normalfont,
  postheadspace = 3pt,
]{definitionstyle}

\declaretheoremstyle[
  spaceabove = 6pt,
  spacebelow = 6pt,
  headfont = \normalfont\itshape,
  notefont = \normalfont,
  bodyfont = \normalfont,
  postheadspace = 3pt,
]{memstyle}

\declaretheoremstyle[
  spaceabove = 6pt,
  spacebelow = 6pt,
  headfont = \normalfont\itshape,
  notefont = \normalfont, % ← шрифт для [1]
  bodyfont = \normalfont,
  headpunct = {.},         % ← точка после номера
  postheadspace = 3pt,
  notebraces = {}{}, % ← без скобок, просто "Пример 1."
]{example}

% ================= СРЕДЫ =================
% Примеры — отдельная нумерация по секции
\declaretheorem[
  numbered = no,        % ← отключаем нумерацию
  title = \uline{Пример}, % ← просто слово "Пример"
  style = example
]{example}

\declaretheorem[
  numbered = no,        % ← отключаем нумерацию
  title = \uline{Понятие}, % ← просто слово "Пример"
  style = example
]{concept}

% Определения — отдельная нумерация по подразделу
\declaretheorem[
  title = Def,
  within = subsection,
  style = definitionstyle
]{definition}

% Теоремы — отдельная нумерация по подразделу
\declaretheorem[
  title = \uline{Теорема},
  within = subsection,
  style = theoremstyle
]{theorem}

% Замечания — отдельная нумерация по подразделу
\declaretheorem[
  title = \uline{Замечание},
  within = subsection,
  style = remarkstyle
]{nota}

% Следствия — отдельная нумерация по подразделу
\declaretheorem[
  title = Следствие,
  within = subsection,
  style = remarkstyle
]{corollary}

% Напоминания — отдельная нумерация по подразделу
\declaretheorem[
  title = \uline{Напоминание},
  within = subsection,
  style = memstyle
]{mem}

% Леммы — отдельная нумерация по подразделу
\declaretheorem[
  title = \underline{Lm},
  within = subsection,
  style = definitionstyle
]{lemma}

% ================= ФОРМАТ НУМЕРАЦИИ =================
% Примеры — простой счётчик
\renewcommand{\theexample}{\arabic{example}}

% Остальные — секция.подсекция.номер
\renewcommand{\thetheorem}{\thesubsection.\arabic{theorem}}
\renewcommand{\thenota}{\thesubsection.\arabic{nota}}
\renewcommand{\thedefinition}{\thesubsection.\arabic{definition}}
\renewcommand{\thelemma}{\thesubsection.\arabic{lemma}}
\renewcommand{\thecorollary}{\thesubsection.\arabic{corollary}}
\renewcommand{\themem}{\thesubsection.\arabic{mem}}


% ================= ДРУГОЕ =================
\addto\captionsrussian{\renewcommand{\proofname}{\(\square\)\nopunct}}
\renewcommand{\qedsymbol}{\(\blacksquare\)}

\addto\captionsrussian{\renewcommand{\figurename}{Рис.}}
\renewcommand{\thefigure}{\thesection.\thesubsection.\arabic{figure}}

\newcommand{\subheader}[1]{\textbf{#1} \medskip}
\newcommand{\subsubheader}[2]{\medskip \textbf{#1} \vspace{6pt} #2 \smallskip}

\pagestyle{fancy}
\fancyhead{}
\fancyfoot{}
\fancyfoot[C]{\thepage/\pageref*{LastPage}}
\fancypagestyle{mainpage}{
  \fancyfoot{}
  \fancyfoot[C]{\url{\githublink}}
}
\renewcommand{\headrulewidth}{0pt}

\setlist[enumerate]{ topsep = 0pt }

\lstset{
  basicstyle = \small\ttfamily,
  tabsize = 2,
  showstringspaces = false,
  numbers = left,
  commentstyle = \color{green},
  morecomment = [l][\color{magenta}]{\#},
  keywordstyle = \color{blue},
  language = C++,
  stringstyle = \color{red}\ttfamily,
  breaklines = true
}
