\subsection{%
  Алгебраические структуры (предмет алгебры)%
}
\begin{definition}
	Алгебраическая структура --- это множество,
	с определенной на нем операциями и их свойствами.
\end{definition}
\begin{definition}
	Алгебраическая группа ($G$) --- множество $G$: \underline{$+$} --- любая операция
	\begin{enumerate}[topsep=5pt]
		\item \(\forall a, b \in \Group \mid a + b \in \Group\) --- замкнутость
		\item $\forall a,b,c \in \Group: a + (b + c) = (a + b) + c$ --- ассоциативность
		\item $\exists \, \theta \, \in \Group \mid \forall a \in \Group: a + \theta = \theta + a = a$ --- наличие нейтрального элемента
		\item $\forall a \in \Group \hspace{0.8em} \existsone \, a' \in \Group \mid a + a' = \theta$ --- наличие обратного элемента
	\end{enumerate}
\end{definition}
\begin{nota}
	Если $\oplus$, то группа --- аддитивная, $\theta$ --- ноль ($\mathbbold{0}$), а $a' = -a$\\[1mm]
	\hspace*{7.6em} Если $\otimes$, то группа --- мультипликативная, $\theta$ --- единица ($\mathbbold{1}$), а $a' = a^{-1}$ (обратный элемент)\\[1mm]
	\noindent\makebox[\linewidth][c]{\hspace{-8.5cm}%
		\begin{smallimportant}{0.25\linewidth}
		$\theta$ --- нейтральный\\
		$a'$ --- обратный элемент
		\end{smallimportant}
		}
\end{nota}
\begin{nota}
	Если к определению группы добавить $a \times b = b \times a$,
	то группа называется \underline{абелевой} или \underline{коммутативной}.
\end{nota}
\begin{definition}
  Кольцо это коммутативная аддитивная группа, в которой

  \begin{enumerate}
  \item
    Определено умножение.

  \item
    Относительно этого умножения выполняется дистрибутивность\\ \(a + b \cdot c =
    a \cdot c + b \cdot c\) и \(c \cdot (a + b) = c \cdot a + c \cdot b\) (т.к.
    коммутативность для умножения не гарантирована).
  \end{enumerate}
\end{definition}

\begin{definition}
  Если кольцо обладает свойством коммутативности относительно умножения
  \(\forall a, b \in \Group \mid a \cdot b = b \cdot a\), то оно называется
  коммутативным кольцом.
\end{definition}
\begin{definition}
  Поле ($\Field$) это коммутативное ассоциативное кольцо, в котором

  \begin{enumerate}
  \item
    Есть нейтральный элемент по умножению \(\exists \theta \mid \forall a \in
    \Field \mid a \cdot \theta = a\).
  \item
    Для любого ненулевого элемента существует обратный элемент по умножению
    \(\forall a \in \Field \mid \exists a^{-1} \mid a \cdot a^{-1} =
    \theta\).
  \end{enumerate}
\end{definition}
