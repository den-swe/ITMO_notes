\subsection{%
  Аксиоматическое строение%
}
\textbf{Евклидова геометрия:}
\smallskip
	\begin{enumerate}
		\item Точка
		\item Прямая
		\item Плоскость
	\end{enumerate}
\bigskip
\textbf{Определение в математике (обычно):}
\smallskip
\begin{example}[1]

	\stackunder{Прямоугольник}{\footnotesize (Определяемое понятие)}— это
	\stackunder{параллелограмм}{\footnotesize (родовое понятие)}, у которого
	\stackunder{есть прямой угол}{\footnotesize (видовое свойство)}
\end{example}
\smallskip
\begin{example}[2]
	\hspace{2mm}Через две точи проходит одна и только одна прямая — \underline{редукция}
\end{example}
\smallskip
\begin{example}[3]
	\hspace{2mm}0! = 1 — \underline{данность}
\end{example}
\medskip
\textbf{Виды определений:}
\begin{enumerate}[topsep=5pt]
	\item Редукция
    \item Аксиоматическое
    \item Дополнение к основному
\end{enumerate}
\smallskip
\begin{nota}
	\underline{Математическое высказывание} — утверждение, допускающее проверку на истинность.\\
	Определение - НЕ математическое высказывание!
\end{nota}
\medskip
\begin{theorem}
Виды, структура
\end{theorem}
\smallskip
"Если $A$, то $B$" \hspace{2mm} или
\stackunder{$A$}{\footnotesize (Условие)}
$\implies$
\stackunder{$B$}{\footnotesize (Заключение)}\\[2mm]
"Тогда и только тогда, когда" \hspace{0.1mm} $\Rarr{}$ $A \Leftrightarrow B$\\[3mm]
Необходимые и достаточные условия:\\[1mm]
\stackunder{Мыслю}{\footnotesize (достаточное условие)} —
\stackunder{существую}{\footnotesize (необходимое условие)}\\[2mm]
\underline{Критерий} — необходимое и достаточное условия.

